\documentclass[a4paper, 9pt]{extletter} 
\usepackage{nopageno}
\usepackage{phonenumbers}
\usepackage{tabularx}
\usepackage{tabularx,array, lipsum}
\usepackage{cellspace}
\usepackage{hyperref}
\usepackage{xcolor}
\usepackage{setspace}
\usepackage{enumitem}

\newenvironment{myTextfield}


\setlength\cellspacetoplimit{6pt}
\setlength\cellspacebottomlimit{6pt}
\addparagraphcolumntypes{X}
\usepackage[ddmmyyyy]{datetime}
\renewcommand{\dateseparator}{.}
\usepackage[paper=a4paper, left=15mm, right=20mm, top=20mm, bottom=15mm]{geometry}

\newcolumntype{R}{>{\raggedright\arraybackslash}X}%


\begin{document} 
\begin{Form}

\begin{flushright}
	Düsseldorf, den \today\\
	(Ilhan) \phonenumber[area-code-sep=brackets]{0211 - 81 - 12615} \\
% 	(Häser) \phonenumber[area-code-sep=brackets]{0211 - 81 - 15841} \\
\end{flushright}
Zentrum/Institut/Klinik
\begin{enumerate}
   \item Über die Vorsitzenden der Komission zur Verteilung der Stunden für stud. und wiss. Hilfskräfte\footnote{siehe Seite 2 \--Anschrift\--}
   \item Über die Dekanin/den Dekan der Medizinischen Fakultät \\ (\textbf{Ziffer 1. und 2. entfällt bei Finanzierung über Drittmittel})
   \item An die Personalabteilung des Universitätsklinikums Düsseldorf 
\end{enumerate}

\underline{Betrifft:}



\begin{center}	
	\begingroup
	\setlength{\tabcolsep}{1em} 
	\begin{tabular}{l c c c }	  
	  Antrag auf  & 	  
		\ChoiceMenu[combo,name=typeEmp1,height=0.1cm,width=2.5cm,borderwidth=0,backgroundcolor={0.95 0.95 0.95}]{}{Einstellung,Weiterbeschäftigung,Wiedereinstellung}
	  & 
	  	einer 
	  &	  
	  	\ChoiceMenu[combo,name=typeEmp2,width=2.5cm,borderwidth=0,backgroundcolor={0.95 0.95 0.95}]{}{stud. Hilfskraft,wiss. Hilfskraft} \\
	\end{tabular} 
	 
	 
	\begin{tabular}{l c c c }	
	  	für die Zeit vom 
 	&
 		\TextField[name=timeFrom,align=1,bordercolor=,backgroundcolor={0.95 0.95 0.95}]{}  
 	&
 	  	bis 
 	&
 		\TextField[name=timeTo,align=1,bordercolor=,backgroundcolor={0.95 0.95 0.95}]{}  
	\end{tabular}
	 
	\endgroup	 
\end{center}



mit einer durchschnittlichen wöchentlichen Arbeitszeit von \textbf{8} Stunden (max. 17 Std. / min 3 Std.)
Gesamtstundenanzahl: \textbf{32} Stunden 




\begin{center}
\begin{tabularx}{\columnwidth}{|X|R|}
  \hline
  \textsf{Name:} \hspace{3.5mm} \TextField[name=vname,width=6.5cm,value={Max},bordercolor=,backgroundcolor={0.95 0.95 0.95}]{}
  &
  \textsf{Nachname:} \TextField[name=nname,width=5cm,value={Mustermann},bordercolor=,backgroundcolor={0.95 0.95 0.95}]{} \\
  \hline
\end{tabularx}
\begin{tabularx}{\columnwidth}{|X|R|}
  \textsf{Anschrift:} \TextField[name=adresse,width=6.5cm,value={Hauptstrasse 31. 40211 Düsseldorf},bordercolor=,backgroundcolor={0.95 0.95 0.95}]{}
  &
  \textsf{Telefon:} \hspace{3.1mm} \TextField[name=telefon,width=4cm,value={0176123456},bordercolor=,backgroundcolor={0.95 0.95 0.95}]{} \\
  \hline
\end{tabularx}
\end{center}


für die Lehrveranstaltung: \hspace{2.4mm} \ChoiceMenu[combo,name=kursName,borderwidth=0,width=8.5cm,bordercolor=,backgroundcolor={0.95 0.95 0.95}]{}{(Ilhan) \textbf{Makroskopisch Anatomischer Kurs},(Häser) \textbf{Mikroskopisch Anatomischer Kurs}}

\begin{myTextfield}
	\renewcommand*{\LayoutTextField}[2]{\makebox[0.6em][l]{#1}\raisebox{\baselineskip}{\raisebox{-\height}{#2}}}
	für Dienstleistungsaufgaben: \TextField[multiline,name=aBeschr,bordercolor=,width=0.5\textwidth,height=1cm,value={},backgroundcolor={0.95 0.95 0.95}]{}
\end{myTextfield}
\hspace{5.2cm} (kurze Angaben zur zu verrichtenden Tätigkeit)

\textbf{Die stud./wiss. Hilfskraft wird:}

\begin{enumerate}
  \renewcommand{\arraystretch}{0.5}
  \item umgang mit Stoffen nach der Gefahrstoffverordnung haben \hfill \CheckBox[name=j1,width=0.4cm,height=0.4cm,bordercolor=,backgroundcolor={0.95 0.95 0.95}]{Ja} \CheckBox[name=n1,width=0.4cm,height=0.4cm,bordercolor=,backgroundcolor={0.95 0.95 0.95}]{Nein}
  \item Röntgen-/Strahlenschutzbereich eingesetzt \hfill \CheckBox[name=j2,width=0.4cm,height=0.4cm,bordercolor=,backgroundcolor={0.95 0.95 0.95}]{Ja} \CheckBox[name=n2,width=0.4cm,height=0.4cm,bordercolor=,backgroundcolor={0.95 0.95 0.95}]{Nein}
  \item Infektiösem- oder Infektionsverdächtigem zu tun haben \hfill \CheckBox[name=j3,width=0.4cm,height=0.4cm,bordercolor=,backgroundcolor={0.95 0.95 0.95}]{Ja} \CheckBox[name=n3,width=0.4cm,height=0.4cm,bordercolor=,backgroundcolor={0.95 0.95 0.95}]{Nein}
  \item im Laborbereich eingesetzt \hfill \CheckBox[name=j4,width=0.4cm,height=0.4cm,bordercolor=,backgroundcolor={0.95 0.95 0.95}]{Ja} \CheckBox[name=n4,width=0.4cm,height=0.4cm,bordercolor=,backgroundcolor={0.95 0.95 0.95}]{Nein}
  \item in Sektionsräumen tätig sein \hfill \CheckBox[name=j5,width=0.4cm,height=0.4cm,bordercolor=,backgroundcolor={0.95 0.95 0.95}]{Ja} \CheckBox[name=n5,width=0.4cm,height=0.4cm,bordercolor=,backgroundcolor={0.95 0.95 0.95}]{Nein}
  \item in der stationären-/ambulanten Krankenversorgung eingesetzt \hfill \CheckBox[name=j6,width=0.4cm,height=0.4cm,bordercolor=,backgroundcolor={0.95 0.95 0.95}]{Ja} \CheckBox[name=n6,width=0.4cm,height=0.4cm,bordercolor=,backgroundcolor={0.95 0.95 0.95}]{Nein}
\end{enumerate}



\CheckBox[name=weekdays,width=0.4cm,height=0.4cm,bordercolor=,backgroundcolor={0.95 0.95 0.95}]{} Beschäftigung erfolgt an folgenden Wochentagen:

\begin{tabular}{llll}
	Montag: & \TextField[name=mmVon,align=1,width=1cm,bordercolor=,backgroundcolor={0.95 0.95 0.95}]{von } & \TextField[name=moBis,align=1,width=1cm,bordercolor=,backgroundcolor={0.95 0.95 0.95}]{bis } & Uhr \\
	Dienstag: & \TextField[name=dVon,align=1,width=1cm,bordercolor=,backgroundcolor={0.95 0.95 0.95}]{von } & \TextField[name=diBis,align=1,width=1cm,bordercolor=,backgroundcolor={0.95 0.95 0.95}]{bis } & Uhr \\
	Mittwoch: & \TextField[name=miVon,align=1,width=1cm,bordercolor=,backgroundcolor={0.95 0.95 0.95}]{von } & \TextField[name=miBis,align=1,width=1cm,bordercolor=,backgroundcolor={0.95 0.95 0.95}]{bis } & Uhr \\
	Donnerstag: & \TextField[name=doVon,align=1,width=1cm,bordercolor=,backgroundcolor={0.95 0.95 0.95}]{von } & \TextField[name=doBis,align=1,width=1cm,bordercolor=,backgroundcolor={0.95 0.95 0.95}]{bis } & Uhr \\
	Freitag: & \TextField[name=frVon,align=1,width=1cm,bordercolor=,backgroundcolor={0.95 0.95 0.95}]{von } & \TextField[name=frBis,align=1,width=1cm,bordercolor=,backgroundcolor={0.95 0.95 0.95}]{bis } & Uhr \\
\end{tabular}
	
\CheckBox[name=tutor,width=0.4cm,height=0.4cm,bordercolor=,backgroundcolor={0.95 0.95 0.95}]{} mit Tutorentätigkeit im Umfang von \TextField[name=aBesch,align=1,width=1cm,bordercolor=,backgroundcolor={0.95 0.95 0.95}]{} \hspace{1mm} Stunden wöchentlich. 
	
	
% \begin{tabularx}{\textwidth}{lX}
% 	Für: 
%  	&
% 	\TextField[name=forTut,width=14cm,value={},bordercolor=,backgroundcolor={0.95 0.95 0.95}]{}
%  	\\
% 	
% 	&
% 	(Art des Tutoriums s. §1 Abs. 3 Buchst. a-g der Dienstverträge für szud./wiss. Hilfskräfte)
% \end{tabularx}

Für: \TextField[name=forTut,width=14cm,value={},bordercolor=,backgroundcolor={0.95 0.95 0.95}]{} \\ 
\phantom{Für:} \hspace{1ex}(Art des Tutoriums s. §1 Abs. 3 Buchst. a-g der Dienstverträge für szud./wiss. Hilfskräfte)
	
	
Falls nicht aus Hilfskraftstunden-Kontingent der Med. Fakultät, Finanzierung durch:			
	
\begin{tabular}{clrl}

\CheckBox[name=drittMittel,width=0.4cm,height=0.4cm,bordercolor=,backgroundcolor={0.95 0.95 0.95}]{}\hspace{0.2cm}Drittmittel:
&
\TextField[name=projectName,width=5cm,value={},bordercolor=,backgroundcolor={0.95 0.95 0.95}]{}
&
\CheckBox[name=freiWiss,width=0.4cm,height=0.4cm,bordercolor=,backgroundcolor={0.95 0.95 0.95}]{}\hspace{0.2cm}freie wiss. Stelle:
&
\TextField[name=stellBeschr,width=5cm,value={},bordercolor=,backgroundcolor={0.95 0.95 0.95}]{}

\\

&
\hspace{1cm}(Projektbezeichnung)
&

&
\hspace{1cm}(Stellenbezeichnung)
\end{tabular}

	
	
\end{Form}
\end{document}